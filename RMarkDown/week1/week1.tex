% Options for packages loaded elsewhere
\PassOptionsToPackage{unicode}{hyperref}
\PassOptionsToPackage{hyphens}{url}
%
\documentclass[
]{article}
\usepackage{amsmath,amssymb}
\usepackage{iftex}
\ifPDFTeX
  \usepackage[T1]{fontenc}
  \usepackage[utf8]{inputenc}
  \usepackage{textcomp} % provide euro and other symbols
\else % if luatex or xetex
  \usepackage{unicode-math} % this also loads fontspec
  \defaultfontfeatures{Scale=MatchLowercase}
  \defaultfontfeatures[\rmfamily]{Ligatures=TeX,Scale=1}
\fi
\usepackage{lmodern}
\ifPDFTeX\else
  % xetex/luatex font selection
\fi
% Use upquote if available, for straight quotes in verbatim environments
\IfFileExists{upquote.sty}{\usepackage{upquote}}{}
\IfFileExists{microtype.sty}{% use microtype if available
  \usepackage[]{microtype}
  \UseMicrotypeSet[protrusion]{basicmath} % disable protrusion for tt fonts
}{}
\makeatletter
\@ifundefined{KOMAClassName}{% if non-KOMA class
  \IfFileExists{parskip.sty}{%
    \usepackage{parskip}
  }{% else
    \setlength{\parindent}{0pt}
    \setlength{\parskip}{6pt plus 2pt minus 1pt}}
}{% if KOMA class
  \KOMAoptions{parskip=half}}
\makeatother
\usepackage{xcolor}
\usepackage[margin=1in]{geometry}
\usepackage{color}
\usepackage{fancyvrb}
\newcommand{\VerbBar}{|}
\newcommand{\VERB}{\Verb[commandchars=\\\{\}]}
\DefineVerbatimEnvironment{Highlighting}{Verbatim}{commandchars=\\\{\}}
% Add ',fontsize=\small' for more characters per line
\usepackage{framed}
\definecolor{shadecolor}{RGB}{248,248,248}
\newenvironment{Shaded}{\begin{snugshade}}{\end{snugshade}}
\newcommand{\AlertTok}[1]{\textcolor[rgb]{0.94,0.16,0.16}{#1}}
\newcommand{\AnnotationTok}[1]{\textcolor[rgb]{0.56,0.35,0.01}{\textbf{\textit{#1}}}}
\newcommand{\AttributeTok}[1]{\textcolor[rgb]{0.13,0.29,0.53}{#1}}
\newcommand{\BaseNTok}[1]{\textcolor[rgb]{0.00,0.00,0.81}{#1}}
\newcommand{\BuiltInTok}[1]{#1}
\newcommand{\CharTok}[1]{\textcolor[rgb]{0.31,0.60,0.02}{#1}}
\newcommand{\CommentTok}[1]{\textcolor[rgb]{0.56,0.35,0.01}{\textit{#1}}}
\newcommand{\CommentVarTok}[1]{\textcolor[rgb]{0.56,0.35,0.01}{\textbf{\textit{#1}}}}
\newcommand{\ConstantTok}[1]{\textcolor[rgb]{0.56,0.35,0.01}{#1}}
\newcommand{\ControlFlowTok}[1]{\textcolor[rgb]{0.13,0.29,0.53}{\textbf{#1}}}
\newcommand{\DataTypeTok}[1]{\textcolor[rgb]{0.13,0.29,0.53}{#1}}
\newcommand{\DecValTok}[1]{\textcolor[rgb]{0.00,0.00,0.81}{#1}}
\newcommand{\DocumentationTok}[1]{\textcolor[rgb]{0.56,0.35,0.01}{\textbf{\textit{#1}}}}
\newcommand{\ErrorTok}[1]{\textcolor[rgb]{0.64,0.00,0.00}{\textbf{#1}}}
\newcommand{\ExtensionTok}[1]{#1}
\newcommand{\FloatTok}[1]{\textcolor[rgb]{0.00,0.00,0.81}{#1}}
\newcommand{\FunctionTok}[1]{\textcolor[rgb]{0.13,0.29,0.53}{\textbf{#1}}}
\newcommand{\ImportTok}[1]{#1}
\newcommand{\InformationTok}[1]{\textcolor[rgb]{0.56,0.35,0.01}{\textbf{\textit{#1}}}}
\newcommand{\KeywordTok}[1]{\textcolor[rgb]{0.13,0.29,0.53}{\textbf{#1}}}
\newcommand{\NormalTok}[1]{#1}
\newcommand{\OperatorTok}[1]{\textcolor[rgb]{0.81,0.36,0.00}{\textbf{#1}}}
\newcommand{\OtherTok}[1]{\textcolor[rgb]{0.56,0.35,0.01}{#1}}
\newcommand{\PreprocessorTok}[1]{\textcolor[rgb]{0.56,0.35,0.01}{\textit{#1}}}
\newcommand{\RegionMarkerTok}[1]{#1}
\newcommand{\SpecialCharTok}[1]{\textcolor[rgb]{0.81,0.36,0.00}{\textbf{#1}}}
\newcommand{\SpecialStringTok}[1]{\textcolor[rgb]{0.31,0.60,0.02}{#1}}
\newcommand{\StringTok}[1]{\textcolor[rgb]{0.31,0.60,0.02}{#1}}
\newcommand{\VariableTok}[1]{\textcolor[rgb]{0.00,0.00,0.00}{#1}}
\newcommand{\VerbatimStringTok}[1]{\textcolor[rgb]{0.31,0.60,0.02}{#1}}
\newcommand{\WarningTok}[1]{\textcolor[rgb]{0.56,0.35,0.01}{\textbf{\textit{#1}}}}
\usepackage{graphicx}
\makeatletter
\def\maxwidth{\ifdim\Gin@nat@width>\linewidth\linewidth\else\Gin@nat@width\fi}
\def\maxheight{\ifdim\Gin@nat@height>\textheight\textheight\else\Gin@nat@height\fi}
\makeatother
% Scale images if necessary, so that they will not overflow the page
% margins by default, and it is still possible to overwrite the defaults
% using explicit options in \includegraphics[width, height, ...]{}
\setkeys{Gin}{width=\maxwidth,height=\maxheight,keepaspectratio}
% Set default figure placement to htbp
\makeatletter
\def\fps@figure{htbp}
\makeatother
\setlength{\emergencystretch}{3em} % prevent overfull lines
\providecommand{\tightlist}{%
  \setlength{\itemsep}{0pt}\setlength{\parskip}{0pt}}
\setcounter{secnumdepth}{-\maxdimen} % remove section numbering
\usepackage[linesnumbered,ruled,lined,boxed]{algorithm2e}
\usepackage{amsmath}
\usepackage{amsfonts}
\ifLuaTeX
  \usepackage{selnolig}  % disable illegal ligatures
\fi
\IfFileExists{bookmark.sty}{\usepackage{bookmark}}{\usepackage{hyperref}}
\IfFileExists{xurl.sty}{\usepackage{xurl}}{} % add URL line breaks if available
\urlstyle{same}
\hypersetup{
  pdftitle={week1},
  pdfauthor={Rajesh Kalakoti},
  hidelinks,
  pdfcreator={LaTeX via pandoc}}

\title{week1}
\author{Rajesh Kalakoti}
\date{2023-08-03}

\begin{document}
\maketitle

\begin{itemize}
\tightlist
\item
  Packages

  \begin{itemize}
  \tightlist
  \item
    \href{https://www.r-project.org/nosvn/pandoc/devtools.html}{devtools}
  \item
    \href{https://www.tidyverse.org/packages/}{tidyverse}

    \begin{itemize}
    \tightlist
    \item
      sub-sub-item 1
    \end{itemize}
  \end{itemize}
\end{itemize}

\hypertarget{clustering}{%
\subsection{Clustering}\label{clustering}}

Given a clustering \(C = \{C_1, C_2, \ldots, C_k\}\), we need some
scoring function that evaluates its quality or goodness. This sum of
squared errors scoring function is defined as:
\[ W(C) = \frac{1}{2} \sum_{k=1}^{K} \sum_{i: C(i)=k} \|x_i - \bar{x}_k\|^2 \]

The goal is to find the clustering that minimizes:

\[ C^* = \arg \min_C \{ W(c) \} \]

K-means employs a greedy iterative approach to find a clustering that
minimizes loss function.

\hypertarget{algorithm-13.1-k-means-algorithm}{%
\subsection{Algorithm 13.1: K-means
Algorithm}\label{algorithm-13.1-k-means-algorithm}}

\textbf{K-means} (\emph{D}, \emph{k}, \emph{ε}): 1. Initialize
\(t = 0\). Randomly initialize \(k\) centroids:
\(\mu_{t1}, \mu_{t2}, \ldots, \mu_{tk} \in \mathbb{R}^d\). 3.
\textbf{repeat} 4. \(t \leftarrow t + 1\). \textbf{// Cluster Assignment
Step} 5. \textbf{foreach} \(x_j \in D\) \textbf{do} 6.
\(j^* \leftarrow \arg \min_i \|x_j - \mu_{ti}\|^2\). \textbf{// Assign
\(x_j\) to closest centroid} 7.
\(C_{j^*} \leftarrow C_{j^*} \cup \{x_j\}\). \textbf{// Centroid Update
Step} 8. \textbf{foreach} \(i = 1\) \textbf{to} \(k\) \textbf{do} 9.
\(\mu_{ti} \leftarrow \frac{1}{|C_i|} \sum_{x_j \in C_i} x_j\).
\textbf{until}
\(\sum_{i=1}^k \|\mu_{ti} - \mu_{t-1i}\|^2 \leq \varepsilon\).

\begin{algorithm}
\LinesNumbered % Add line numbers to the algorithm
\caption{K-means Algorithm}
\KwData{$D, k, \varepsilon$}
\KwResult{Result $y$}

\textbf{K-means}($D, k, \varepsilon$) {
  
  $t \leftarrow 0$\;
  Randomly initialize $k$ centroids: $\mu_{1}^{t}, \mu_{2}^{t}, \ldots, \mu_{n}^{t} \in \mathbb{R}^d$\;
  \Repeat{termination condition}{
    $t \leftarrow t+1$\;
    /* Cluster assignment step */
    
    \For{$x_j \in D$}{
      $j^* \leftarrow \text{argmin}_i \{||x_j - \mu_{i}^t||^2 \}$\;
      /* assign $x_j$ to closest centroid */
      
      $C_{j^*} \leftarrow C_{j^*} \cup \{x_j\}$\;
    }
  }
  Perform additional steps\;
  \For{$i = 1$ to $n$}{
    Update $y \leftarrow y \times x$\;
  }
  \textbf{return} $y$\;
}
\end{algorithm}

Note that the \texttt{echo\ =\ FALSE} parameter was added to the code
chunk to prevent printing of the R code that generated the plot.

\begin{Shaded}
\begin{Highlighting}[]
\NormalTok{?entropy}
\end{Highlighting}
\end{Shaded}

\begin{verbatim}
## No documentation for 'entropy' in specified packages and libraries:
## you could try '??entropy'
\end{verbatim}

\end{document}
